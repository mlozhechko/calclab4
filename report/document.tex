\documentclass[12pt]{article}
% Эта строка — комментарий, она не будет показана в выходном файле
\usepackage{ucs}
\usepackage[utf8x]{inputenc} % Включаем поддержку UTF8
\usepackage[russian]{babel}  % Включаем пакет для поддержки русского языка
\usepackage{amsmath, amsthm, amssymb, amsfonts}
\usepackage{multirow}
\usepackage{pbox}
\usepackage[showframe=false]{geometry}
\usepackage{changepage}

\date{}
\author{}

\begin{document}
\subsubsection*{Вопросы}
\begin{enumerate}
	\item Результаты вычислений для $\varepsilon=10^{-2}, 10^{-4}, 10^{-6}$
	\item Результаты работы метода обратных итераций. Критерий останова для данного метода.
	\item Для модификации метода обратных итераций, гарантирует ли нахождение полного собственного базиса выбор в качестве начальных приближений векторов, составляющих единичную матрицу.
\end{enumerate}



\subsubsection*{Разбор}
\begin{enumerate}	
\item Результаты вычислений:
	\begin{enumerate}
	\item $\varepsilon=10^{-2}$\\
	\begin{adjustwidth}{-3cm}{}
	\begin{tabular}{|l|l|l|l|}
		\hline
		\multirow{2}{*}{} & \multirow{2}{*}{} & \multicolumn{2}{l|}{QR-разложение} \\ \cline{3-4} 
		&                   &   \pbox{20cm}{С приведением\\ к форме Хессенберга}      &  \pbox{20cm}{Без приведения\\ к форме Хессенберга}        \\ \hline
		\multirow{2}{*}{Сдвиг} &    $-$               &  Число итер: 19 Число мульт. опер: 3018   &        Число итер: 33 Число мульт. опер: 6430   \\ \cline{2-4} 
		&    $+$              &     Число итер: 6 Число мульт. опер: 602     &        Число итер: 5 Число мульт. опер: 772   \\ \hline
	\end{tabular}
	\end{adjustwidth}
	\item $\varepsilon=10^{-4}$\\
	\begin{adjustwidth}{-3cm}{}
	\begin{tabular}{|l|l|l|l|}
		\hline
		\multirow{2}{*}{} & \multirow{2}{*}{} & \multicolumn{2}{l|}{QR-разложение} \\ \cline{3-4} 
		&                   &  \pbox{20cm}{С приведением\\ к форме Хессенберга}      &  \pbox{20cm}{Без приведения\\ к форме Хессенберга}        \\ \hline
		\multirow{2}{*}{Сдвиг} &    $-$               &     Число итер: 40 Число мульт. опер: 5790 &        Число итер: 55 Число мульт. опер:  9844  \\ \cline{2-4} 
		&    $+$              &         Число итер: 7 Число мульт. опер:  707 &      Число итер: 5 Число мульт. опер:772      \\ \hline
	\end{tabular}
	\end{adjustwidth}
	\item $\varepsilon=10^{-6}$\\
	\begin{adjustwidth}{-3cm}{}
	\begin{tabular}{|l|l|l|l|}
		\hline
		\multirow{2}{*}{} & \multirow{2}{*}{} & \multicolumn{2}{l|}{QR-разложение} \\ \cline{3-4} 
		&                   &   \pbox{20cm}{С приведением\\ к форме Хессенберга}      &  \pbox{20cm}{Без приведения\\ к форме Хессенберга}        \\ \hline
		\multirow{2}{*}{Сдвиг} &    $-$               &    Число итер: 62 Число мульт. опер: 8694 &           Число итер: 76 Число мульт. опер: 12310 \\ \cline{2-4} 
		&    $+$              &         Число итер: 8  Число мульт. опер: 733   &      Число итер: 7 Число мульт. опер:  1066    \\ \hline
	\end{tabular}
	\end{adjustwidth}
	\end{enumerate}
\item Результат работы метода обратных итераций

\end{enumerate}

\end{document}