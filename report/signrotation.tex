\documentclass[12pt]{article}
% Эта строка — комментарий, она не будет показана в выходном файле
\usepackage{ucs}
\usepackage[utf8x]{inputenc} % Включаем поддержку UTF8
\usepackage[russian]{babel}  % Включаем пакет для поддержки русского языка
\usepackage{amsmath, amsthm, amssymb, amsfonts}
\usepackage{multirow}
\usepackage{pbox}
\usepackage[showframe=false]{geometry}
\usepackage{changepage}

\date{}
\author{}

\begin{document}
\subsubsection*{Вопросы}
\begin{enumerate}
	\item Исследовать чередование знаков в методе обратной итерации
\end{enumerate}



\subsubsection*{Разбор}
\begin{enumerate}	
\item На практике наблюдается монотонная сходимость для собственных значений меньших по модулю: $-0.06; 36.9$ и меняется знак для: $-45.87; 88.5004$.

Объяснить это можно следующим образом. Воспользуемся выкладками [1, с. 228].
$$ y^{(1)} \approx \frac{c_{j}}{\lambda_{j}-\lambda_{j}^{*}} e_{j} $$
$$ x^{(0)}=\sum_{i=1}^{m} c_{i} e_{i} $$

$e_{1},e_{2}, \ldots, e_{m}$ -- собственные вектора
Следовательно
$$ y^{(k + 1)} \approx \frac{c_{j}^{k}}{\lambda_{j}-\lambda_{j}^{*}} e_{j} $$
$$ x^{(k)}=\sum_{i=1}^{m} c_{i}^{k} e_{i} $$

Таким образом чередование знака зависит от погрешности $\lambda_{j}^{*}$. В случае
если $\lambda_{j}-\lambda_{j}^{*} > 0$ чередования не будет, в обратном случае знаки будут чередоваться.
\end{enumerate}

\begin{thebibliography}{3}
\bibitem{amosov}
Амосов А.А., Дубинский Ю.А., Копченова Н.В. Вычислительные методы для инженеров: Учеб. пособие. — М.: Высш. шк., 1994. — 544 с.
	
\end{thebibliography}

\end{document}